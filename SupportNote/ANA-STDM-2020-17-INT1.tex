%-------------------------------------------------------------------------------
% This file provides a skeleton ATLAS note.
% \pdfinclusioncopyfonts=1
% This command may be needed in order to get \ell in PDF plots to appear. Found in
% https://tex.stackexchange.com/questions/322010/pdflatex-glyph-undefined-symbols-disappear-from-included-pdf
%-------------------------------------------------------------------------------
% Specify where ATLAS LaTeX style files can be found.
\newcommand*{\ATLASLATEXPATH}{latex/}
% Use this variant if the files are in a central location, e.g. $HOME/texmf.
% \newcommand*{\ATLASLATEXPATH}{}
%-------------------------------------------------------------------------------
\documentclass[NOTE, atlasdraft=true, texlive=2016, UKenglish]{\ATLASLATEXPATH atlasdoc}
% The language of the document must be set: usually UKenglish or USenglish.
% british and american also work!
% Commonly used options:
%  atlasdraft=true|false This document is an ATLAS draft.
%  texlive=YYYY          Specify TeX Live version (2016 is default).
%  coverpage             Create ATLAS draft cover page for collaboration circulation.
%                        See atlas-draft-cover.tex for a list of variables that should be defined.
%  cernpreprint          Create front page for a CERN preprint.
%                        See atlas-preprint-cover.tex for a list of variables that should be defined.
%  NOTE                  The document is an ATLAS note (draft).
%  PAPER                 The document is an ATLAS paper (draft).
%  CONF                  The document is a CONF note (draft).
%  PUB                   The document is a PUB note (draft).
%  BOOK                  The document is of book form, like an LOI or TDR (draft)
%  txfonts=true|false    Use txfonts rather than the default newtx
%  paper=a4|letter       Set paper size to A4 (default) or letter.

%-------------------------------------------------------------------------------
% Extra packages:
\usepackage{\ATLASLATEXPATH atlaspackage}
% Commonly used options:
%  biblatex=true|false   Use biblatex (default) or bibtex for the bibliography.
%  backend=bibtex        Use the bibtex backend rather than biber.
%  subfigure|subfig|subcaption  to use one of these packages for figures in figures.
%  minimal               Minimal set of packages.
%  default               Standard set of packages.
%  full                  Full set of packages.
%-------------------------------------------------------------------------------
% Style file with biblatex options for ATLAS documents.
\usepackage{\ATLASLATEXPATH atlasbiblatex}

% Package for creating list of authors and contributors to the analysis.
\usepackage{\ATLASLATEXPATH atlascontribute}

% Useful macros
\usepackage{\ATLASLATEXPATH atlasphysics}
\usepackage{xfrac}
\usepackage{enumitem}
\DeclareMathOperator*{\argmax}{argmax}
\DeclareMathOperator*{\argmin}{argmin}
% See doc/atlas_physics.pdf for a list of the defined symbols.
% Default options are:
%   true:  journal, misc, particle, unit, xref
%   false: BSM, heppparticle, hepprocess, hion, jetetmiss, math, process, other, texmf
% See the package for details on the options.

% Files with references for use with biblatex.
% Note that biber gives an error if it finds empty bib files.
\addbibresource{ANA-STDM-2020-17-INT1.bib}
\addbibresource{bib/ATLAS.bib}
\addbibresource{bib/CMS.bib}
\addbibresource{bib/ConfNotes.bib}
\addbibresource{bib/PubNotes.bib}

% Paths for figures - do not forget the / at the end of the directory name.
\graphicspath{{logos/}{figures/}}

% Add you own definitions here (file ANA-STDM-2020-17-INT1-defs.sty).
\usepackage{ANA-STDM-2020-17-INT1-defs}

%-------------------------------------------------------------------------------
% Generic document information
%-------------------------------------------------------------------------------

% Title, abstract and document
%-------------------------------------------------------------------------------
% This file contains the title, author and abstract.
% It also contains all relevant document numbers used for an ATLAS note.
%-------------------------------------------------------------------------------

% Title
\AtlasTitle{A simultaneous measurement of all charged particle properties in $Z$+jets events in $\sqrt{s}=13$ TeV $pp$ collisions using the ATLAS detector}

% Draft version:
% Should be 1.0 for the first circulation, and 2.0 for the second circulation.
% If given, adds draft version on front page, a 'DRAFT' box on top of each other page,
% and line numbers.
% Comment or remove in final version.
\AtlasVersion{0.2}

%https://atlas-glance.cern.ch/atlas/analysis/analyses/details.php?id=5225

% Abstract - % directly after { is important for correct indentation
\AtlasAbstract{%
This support note describes a measurement of all charged particle properties in $Z$+jets events using the full Run 2 $pp$ collision dataset recorded by the ATLAS detector.  In particular, this is the first analysis to unfold data without binning and using a variable-length and high-dimensional phase space.  The phase space space is defined by the kinematic properties of all charged particles in boosted $Z\to\mu\mu$ events within a fiducial volume. The unfolding is implemented using the recently proposed OmniFold technique, which uses iteratively trained neural networks.  The analysis itself is broken into two part and this note serves as the reference for both parts.  The first part considers the case of a fixed number of observables (MultiFold) while the second part considers the case of all charged particles (OmniFold).
}

% Author - this does not work with revtex (add it after \begin{document})
%\author{The ATLAS Collaboration}

% Authors and list of contributors to the analysis
% \AtlasAuthorContributor also adds the name to the author list
% Include package latex/atlascontribute to use this
% Use authblk package if there are multiple authors, which is included by latex/atlascontribute
 \usepackage{authblk}
% Use the following 3 lines to have all institutes on one line
% \makeatletter
% \renewcommand\AB@affilsepx{, \protect\Affilfont}
% \makeatother
% \renewcommand\Authands{, } % avoid ``. and'' for last author
% \renewcommand\Affilfont{\itshape\small} % affiliation formatting
\AtlasAuthorContributor{Dag Gillberg}{a}{Ntuple production, analysis strategy, MC setup, advising Sandeep and Laura.}
\AtlasAuthorContributor{Sandeep Kaur}{a}{Data stability, jet composition studies, theory systematics.}
%\AtlasAuthorContributor{Matt LeBlanc}{b}{Alternative observables.}
%\AtlasAuthorContributor{Eric Metodiev}{e}{Short Term Associate.  Initial studies on dataset size.}
\AtlasAuthorContributor{Laura Miller}{a}{Ntuple production, MC setup, analysis strategy, event selection, data-MC comparison, systematics, statistical uncertainty using boot strapping.}
\AtlasAuthorContributor{Ben Nachman}{b}{Unfolding mechanics, event selection cross-check, analysis strategy, advising Adi.}
%\AtlasAuthorContributor{Jennifer Roloff}{d}{Track jet studies.}
\AtlasAuthorContributor{Adi Suresh}{b,c}{Unfolding setup, stress-tests, and results.}
% \AtlasAuthorContributor{Second AtlasAuthorContributor}{b}{Author's contribution.}
% \AtlasAuthorContributor{Third AtlasAuthorContributor}{a}{Author's contribution.}
% \AtlasContributor{Fourth AtlasContributor}{Contribution to the analysis.}
% \author[a]{First Author}
% \author[a]{Second Author}
% \author[b]{Third Author}
\affil[a]{Carleton University}
\affil[b]{Lawrence Berkeley National Laboratory}
\affil[c]{University of California, Berkeley}
%\affil[b]{European Center for Nuclear Research (CERN)}
%\affil[d]{Brookhaven National Laboratory (BNL)}
%\affil[e]{Massachusetts Institute of Technology (MIT)}
% \affil[b]{Another Institution}


% If a special author list should be indicated via a link use the following code:
% Include the two lines below if you do not use atlasstyle:
% \usepackage[marginal,hang]{footmisc}
% \setlength{\footnotemargin}{0.5em}
% Use the following lines in all cases:
% \usepackage{authblk}
% \author{The ATLAS Collaboration%
% \thanks{The full author list can be found at:\newline
%   \url{https://atlas.web.cern.ch/Atlas/PUBNOTES/ATL-PHYS-PUB-2016-007/authorlist.pdf}}
% }

% ATLAS reference code, to help ATLAS members to locate the paper
\AtlasRefCode{ANA-STDM-2020-17}

% ATLAS note number. Can be an COM, INT, PUB or CONF note
\AtlasNote{ANA-STDM-2020-17-INT1}

% Author and title for the PDF file
\hypersetup{pdftitle={ATLAS document},pdfauthor={The ATLAS Collaboration}}

%-------------------------------------------------------------------------------
% Content
%-------------------------------------------------------------------------------
\begin{document}

\maketitle

\tableofcontents

% List of contributors - print here or after the Bibliography.
%\PrintAtlasContribute{0.30}
%\clearpage

\clearpage

\section{Executive summary}
\label{sec:exec}

\subsection{Target}

This analysis uses an innovative machine learning method~\cite{1911.09107} to perform an unbinned, variable- and high-dimensional unfolding using $Z$+jets events. As the first search using these new techniques, we limit the phase space to relatively high \pt{} $Z$~bosons in the $Z\to\mu\mu$ channel.   The analysis uses the full Run 2 dataset and targets summer/fall 2021.

\subsection{Context}

While there is no other measurement that is unbinned, variable- and high-dimensional, there are a variety of related binned measurements of specific observables.   These include track-based observables in inclusive $Z$+jets events at $\sqrt{s}=\SI{8}{\TeV}$~\cite{STDM-2015-14} and various measurements of hadronic final states using tracks~\cite{STDM-2018-57,STDM-2017-33,STDM-2017-16}.  None of these measurements are directly comparable to the one presented in this analysis because of the topology and/or the phase space.  However, as we are using a new methodology for the first time, one of the goals of this paper is to compare the method with standard techniques (e.g.\ Iterative Bayesian Unfolding~\cite{DAGOSTINI1995487,1974AJ.....79..745L,Richardson:72}) in addition to presenting the new unbinned data.

\subsection{Contributors}

\PrintAtlasContribute{0.30}

\clearpage

\section{Changelog}

\begin{itemize}
\item Version 0.1: Initial version, to be used for the editorial board request.
\end{itemize}

\clearpage

%-------------------------------------------------------------------------------
\section{Introduction}
\label{sec:intro}
%-------------------------------------------------------------------------------

The goal of the Large Hadron Collider (LHC) is to infer properties of nature at subnuclear length scales.   One strategy for data analysis is to use models like the Standard Model (SM) and its extensions to fit parameter values to data.  This is an effective strategy that has been used to measure Standard Model couplings and masses as well as to constrain the strength of new physics extensions of the SM.  A key limitation of this parametric approach is that the final result cannot be easily used to infer properties of other parameters or reinterpreted in the context of a different model.  An alternative strategy is to correct data for detector effects in order to measure differential cross sections.  These \textit{unfolded} spectra can be reused for a variety of model interpretations.   Unfolding facilitates data preservation for reuse over time and comparisons across experiments.

The most widely used unfolding methods use various forms of regularized matrix inversion~\cite{DAGOSTINI1995487,Hocker:1995kb,Schmitt:2012kp}.  There are four challenges with these approaches that limit their usefulness.   First, the target observables must be specified prior to unfolding and cannot be changed after the measurement.  Similarly, the binning of the observables must be fixed at the start of the measurement.  Due to the binned nature of existing techniques, most measurements are limited to a small number of observables (usually one) to be simultaneously unfolded.  Finally, the existing methods are not able to account for auxiliary features that determine the resolution of the target observables and thus limit the precision of the measurement.

A variety of alternative unfolding method have been proposed to solve a subset of these challenges.   For example, some proposals avoid binning~\cite{Glazov:2017vni,Datta:2018mwd,Lindemann:1995ut,Aslan:2003vu} and others use machine learning to improve various aspects of the measurement precision~\cite{Gagunashvili:2010zw,Glazov:2017vni,Datta:2018mwd}.  Recently, three techniques have been proposed that have the potential to address all of the above challenges: OmniFold~\cite{1911.09107}, Conditional GAN Unfolding (CGU)~\cite{Bellagente:2019uyp}, and Conditional Normalizing Flow Unfolding (CNFU)~\cite{Bellagente:2020piv}.   The OmniFold approach scaffolds a simulation with a neural network to perform high- and variable-dimensional reweighting.  The CGU and CNFU methods use neural networks to generate corrected events given detector-level events.

This analysis uses the OmniFold approach to perform the first high- and variable-dimensional unbinned measurement.  In contrast to the CFU and CNFU methods, OmniFold can naturally account for point-cloud nature of collider events.  Furthermore, OmniFold reduces to the widely studied Iterative Bayesian Unfolding approach~\cite{DAGOSTINI1995487} when the inputs are binned.  Finally, by using reweighting instead of direct generative modeling, the neural networks need to only learn small corrections to the simulation instead of learning the full probability density of the data.   The physical system chosen for the measurement is the hadronic activity in boosted $Z$ boson events.  Leptonically decaying $Z$ bosons can be identified with high purity and efficiency.  The rest of the event can then be studied with little bias from the event selection.  Charged particles are used due to the precision with which they can be measured.  These events have many tens of charged particles each with a momentum and electric charge and thus the target phase space is about 100-dimensional.

This note is organized as follows.  First, Sec.~\ref{sec:samples} introduces the data and simulated event samples used for the analysis.  Then, Sec.~\ref{sec:objects} describes the objects used in the analysis (in particular, charged particles and tracks).  The analysis methodology is disucssed in Sec.~\ref{sec:strategy} and the prescription for systematic uncertainties appears in Sec.~\ref{sec:uncerts}.  Results are presented in Sec.~\ref{sec:result} and the note concludes in Sec.~\ref{sec:conclusion}.

%-------------------------------------------------------------------------------
\section{Event samples}
\label{sec:samples}
\input{4_data_MC}

%-------------------------------------------------------------------------------
\section{Event and object selection}
\label{sec:objects}

\subsection{Event Selection}
\label{sec:selection}

Initial event selection is outlined in Table~\ref{lab:eventsel}. This is primarily designed to remove problematic and background events in favour of the $Z\rightarrow\mu\mu + \text{jets}$ signal process. The various cuts are discussed in more detail below.

\begin{table}[h!]
    \centering
    \begin{tabular}{l|l}
         \hline
    \textbf{Event Selection} & \textbf{Description} \\ \hline
    Good Runs List & Event must be part of GRL \\ \hline
    Event Cleaning & No LAr, tile calorimeter, or tracker errors. Event is complete. \\ \hline
    Trigger & Single muon trigger: \\
    & HLT\_mu20\_iloose\_L1MU15\_OR\_HLT\_mu50 (2015) \\
    & HLT\_mu26\_ivarmedium\_OR\_HLT\_mu50 (2016-18) \\ \hline
    Primary Vertex & $N_{PV}\geq1$ \\ \hline
    Muons & $N_{\text{muons}}\geq2$ \\
          & Opposite charges \\
          & Pass TTVA recommendations for muons \\
          & $81\leq m_{\ell\ell} (\text{GeV})\leq 101$ \\
          & $p_{\text{T},\ell\ell}\geq200$ GeV \\ \hline
    \end{tabular}
    \caption{Overview of the applied event selection.}
    \label{lab:eventsel}
\end{table}

\subsubsection{Event Requirements}
The first requirement is that the event must be a part of the ATLAS Good Run Lists (GRL) [GRL Ref]. These are discussed in Sec.~\ref{sec:samples}.

Secondly, the events must pass event cleaning [Event Cleaning Ref]. This removes events that:
\begin{itemize}
    \item Contain errors in the data collected in the LAr calorimeters (check for error on xAOD::EventInfo::LAr)
    \item Contain errors in the data collected in the tile calorimeters (check for error on xAOD::EventInfo::Tile)
    \item Contain errors in the silicon tracker data (check for error on xAOD::EventInfo::SCT)
    \item Events which are incomplete due to a restart of the trigger, timing and control system (check for event flag in xAOD::EventInfo::Core)
    \item Events which fail the Loose event cleaning requirement (DFCommonJets\_eventClean\_LooseBad). This is also applied to MC samples.
\end{itemize}

Additionally, it is required that the event in question have at least one primary vertex.

\subsubsection{Trigger Requirements}
Events are required to pass an unprescaled single muon trigger, as outlined in Table~\ref{lab:eventsel} [Trigger Ref]. The recommended trigger is dependent on the year of the data (or corresponding Monte Carlo sample).

\subsubsection{Di-Muon System Requirements}
The event is required to have at least two muons of opposite charge, with a dilepton mass between 81 GeV and 101 GeV in order to ensure the muons originate from a $Z$ boson.

Additionally, the muons must pass track-to-vertex association recommendations [TTVA ref] as follows:

\begin{itemize}
    \item $|d_0^{BL}\text{significance}| < $ 3
    \item $|\Delta z_0^{BL}\sin\theta| < 0.5 $mm
\end{itemize}

where $d_0$ and $z_0$ represent the point along the track that is the closest approach to the beamspot.

The large $p_{\text{T},\ell\ell}$ cut of 200 GeV has been chosen to ensure that the balancing jet(s) have a high quantity of charged particles, as these are the primary subject of the analysis.

\subsection{Detector-level Objects}

\paragraph{Muons} are reconstructed using a combination of information from the muon spectrometer (MS) and the inner detector (ID). Occasionally, the ATLAS calorimeters are used to provide additional information for reconstruction. The reconstruction process is described in detail in reference [Muon Ref].
The muons are required to pass medium quality selection using the MuonSelectionTool and the PflowLoose\_VarRad isolation working point [Iso ref] as given by the IsolationTool. The simulated muons are also calibrated in order to correct for $p_\text{T}$ discrepancies between data and simulation [Muon Ref].

\paragraph{Tracks} are reconstructed from charged-particle hits within the silicon- and straw-tube based inner tracking detectors. They are required to pass the Loose quality working point and to be associated with the primary vertex of the event using the Tight track-to-vertex association.  The primary vertex is defined  as the reconstructed vertex with the highest sum of associated track $\pt^2$.  Tracks are required to have a $\pt>500$~\MeV~to be included in this measurement.  The track $\eta$ and $\phi$ coordinates come from the five-parameter track fit to $(\eta,\phi,q/p,d_0,z_0)$, and thus correspond to the track coordinates at the origin.

\paragraph{Jets} are reconstructed using the anti-$k_t$ algorithm [antikt Ref] with a radius parameter $R=0.4$. Particle flow objects [] are used as input. The jets are calibrated using the recommended procedure for small-R jets, this includes: MCJES calibration, global sequential calibration, and in-situ calibration.
Jets are required to have a calibrated $p_\text{T} > 10~\text{GeV}$ and rapidity $|y| < 4.4$.

\subsubsection{Overlap Removal}

There is a possibility that an electron may be misidentified as a jet during reconstruction, so in cases where the jet is found to be within $\Delta R < 0.2$ of a reconstructed electron, the jet is removed.
Additionally, a specialized overlap removal procedure must be applied between muons and jets due to a bug in the particle flow algorithm which results in secondary muon tracks being included in the particle flow objects.

A summary of the detector level event selection can be found in table~\ref{tab:ObjCuts}.


\begin{table}[h!]
    \centering
    \begin{tabular}{l|l}
    \hline
     \textbf{Object} & \textbf{Additional Selection Criteria} \\ \hline
     Calibrated Muons & Medium Quality, pass isolation: PflowLoose\_VarRad \\ \hline
     Calibrated Jets (AntiKt4EMPFlowJets) & $\Delta R_{e,jet} > 0.2$ \\
      & Muon-PFlow jet overlap removal \\ \hline
     Tracks & Loose Quality \\
      & Tight TTVA \\
      & $p_{\text{T}} > 500$ MeV \\ \hline
    \end{tabular}
    \caption{Object level cuts}
    \label{tab:ObjCuts}
\end{table}

\subsection{Particle-level Objects}

Stable charged particles ($c\tau$ > 10 mm) are used to define the analog of tracks at particle-level. Charged-particles are required to have $\pt > 500$~\MeV~and $|\eta|<2.5$.
Muons are dressed using all photons within a radius of $R=0.1$ and identical kinematic cuts are applied to those used at the detector-level.
TruthWZJets are utilized at the particle-level, which are constructed utilizing the anti-$k_t$ algorithm with $R=0.4$ from all particle 4-vectors except for prompt leptons from $W$, $Z$, Higgs, and $\tau$ decay, and photons within a cone of $R=0.1$ around the prompt lepton.
Objects at particle-level are summarized in table~\ref{tab:PLObjCuts}.

\begin{table}[h!]
    \centering
    \begin{tabular}{l|l}
    \hline
    \textbf{Object} & \textbf{Acceptance} \\ \hline
    Dressed Muons & $p_\text{T} \geq 25$ GeV, $|\eta| \leq 2.4$ \\\hline
    Jets (TruthWZJets) & $p_\text{T}\geq 10$ GeV, $|y|\leq4.4$ \\\hline
    Charged Hadrons & $p_\text{T} \geq 500$ MeV, $|\eta| \leq 2.5$  \\ \hline
    Phase Space & At least two oppositely charged muons \\
    & $81\leq m_{\ell\ell} \text{ (GeV)}\leq101$ \\
    & $p_{\text{T},\ell\ell}\geq200$ GeV \\ \hline
    \end{tabular}
    \caption{Fiducial object cuts}
    \label{tab:PLObjCuts}
\end{table}

\subsection{Corrections to Monte Carlo Samples}
\label{subsec:MCCorr}

\subsection{Data Stability Tests}
To check stability of data used for this analysis, rate for each year is calculated by dividing events to their corresponding luminosities. Luminosity per run is combined until summed up luminosities become either 2 fb$^{-1}$ or greater than this. Figure~\ref{fig:DataStability} depicts rate for year 2015 to 2018, where each time-ordered data bin contains 2 fb$^{-1}$ luminosity. The first bin shows rate for year 2015, bins 2--16 are for year 2016, bins 17--37 and 38--64 represent rate for year 2017 and 2018 respectively. Further, to consider those residual runs having combined luminosities smaller than 2 fb$^{-1}$, the last bin for each year is merged with the previous bin. The calculated average rate for year 2015--2018 is 2845 $\pm$ 4 and shows consistentency for each year. Figure~\ref{fig:DataStability} shows the stability of data for each year used in this analysis.
\begin{figure}[h!]
\centering
\includegraphics[width=0.75\textwidth]{figures/DataStability.pdf}
\caption{Events per fb$^{-1}$ for year 2015 to 2018, where each time-ordered data bin contains 2 fb$^{-1}$ luminosity.}
\label{fig:DataStability}
\end{figure}

\subsection{Data and Monte Carlo Comparison}


%-------------------------------------------------------------------------------
\section{Methodology}
\label{sec:strategy}

\subsection{Unfolding procedures}

\subsubsection{Introduction}

Unfolding is the process of removing detector effects from data in order to infer the properties of particle-level spectra (see~\cite{Cowan:2002in,Blobel:2203257,doi:10.1002/9783527653416.ch6,Balasubramanian:2019itp} for reviews).  In other fields, this is often called \textit{deconvolution}, since one can think of detector effects as the convolution of a noise function with the spectrum of interest.  Unfolding needs to correct for many effects:

\begin{enumerate}[label={(\arabic*)}]
\item Acceptance and efficiency: particles produced may not be measured.
\item Detector noise: particles measured may not be from real particles, at least not the desired particles (e.g. fakes) within a fiducial volume (migrating into acceptance).
\item Background processes: if one wants to measure the differential cross section of a particular process, then you may want to subtract the contributions from background processes.
\item Combinatorics: if there are $n$ particles and you want to measure the properties of a particular order (e.g. leading $p_T$), the detector effects can change the order.
\item Detector distortions: the detector response introduces bias and resolution effects.
\end{enumerate}

A well-structured measurement will be dominated by (5).  It is possible to setup a measurement so that (4) is not relevant (e.g. operating at the level of events as sets instead of using individual objects).  By using a pure (e.g. $Z$+jets) or inclusive (e.g. dijets) event selection, (3) can be made irrelevant.  It is also possible to mitigate (1) and (2) by performing the unfolding in a bigger phase space than is eventually used for the final measurement.  In our case, we will also reduce these effects by using the highly efficient $Z$ to select events and then using (mostly) the hadronic recoil for the measurement.

Corrections are derived using simulations, by creating a match between particle-level events and detector-level events.  Until now, all measurements have been performed using binned data.   In this case, detector effects can be modeled by a set of linear equations:

\begin{align}
\label{eq:foldingequation}
\left(\textbf{R}\cdot (\textbf{t}\odot \textbf{c})\right)\odot 1/\textbf{f}+\textbf{b}=\textbf{d}\,,
\end{align}
%
where bold letters denote vectors or matrices, $\cdot$ is the usual matrix product, $\odot$ is the Hadamard (component-wise) product, and the division is defined component-wise. The symbol $\textbf{t}$ is the particle-level distribution, $\textbf{d}$ is the detector-level distribution, $\textbf{b}$ is the background detector-level distribution, and $\textbf{R}$ is the response matrix.  The correction factors $\textbf{c}$ represent the fraction of particle-level events that also pass the detector-level selection and the fake factors $\textbf{f}$ represent the fraction of detector-level events that have a corresponding particle-level event that passes the selection.  The solution to Eq.~\ref{eq:foldingequation} is given by

\begin{align}
\label{eq:foldingequation}
\textbf{t}_\text{measured} = \textbf{R}^{-1}\left( (\textbf{d}-\textbf{b})\odot \textbf{f}\right)\odot 1/\textbf{c}.
\end{align}

Many of the of the common unfolding methods follow the form in Eq.~\ref{eq:foldingequation}, but vary in their estimation of $\textbf{R}^{-1}$.  For a variety of reasons, it is advantageous to not directly invert the response matrix.  In particular, \textbf{R} may not be invertible because it is singular or not even a square matrix.  The simple matrix inverse is also susceptible to oscillations from significant off-diagonal components.

One of the most widely used unfolding methods is the Iterative Bayesian Unfolding (IBU) technique\footnote{In other fields, this is called Lucy-Richardson deconvolution~\cite{1974AJ.....79..745L,Richardson:72}.}.   For simplicity, assume that $\textbf{b}=\textbf{0}$ and $\textbf{f}=\textbf{c}=\textbf{1}$.  When this is not the case, these are corrected for in the binned case using Eq.~\ref{eq:foldingequation}.   The IBU method then proceeds as follows:

\begin{align}
\textbf{t}^{(n)}&=\sum_j \Pr(t_i|d_j) d_j \\
&=\sum_j \frac{\Pr(d_j|t_i)\Pr^{(n)}(t_i)}{\sum_i \Pr(d_j|t_i)\Pr^{(n)}(t_i)} d_j \\
&=\sum_j \frac{R_{ji}\Pr^{(n)}(t_i}{\sum_i R_{ji}\Pr^{(n)}(t_i)} d_j.
\end{align}
%
Typically, one choses the prior $\Pr^{(n)}(t_i)=t_i^{(n-1)}$ and $\Pr^{(0)}(t_i)=t_i$ (from simulation), assuming $\textbf{t}$ is normalized.  It has been shown that as $n\rightarrow\infty$, the IBU estimate approaches the maximum likelihood estimator~\cite{shepp1982maximum}.  Typically, $n\sim 3$ as a form of regularization to balance amplifying statistical fluctuations with mitigating systematic uncertainties from the prior dependence.

\subsubsection{OmniFold}
\label{subsec:omnifold}

IBU and other standard unfolding methods face three challenges.  First, they require the data to be binned.  This binning must be chosen ahead of time and is often chosen manually.  Second, only a small number of observables (or a small number of bins for a given number of observables) can be unfolded simultaneously in order to keep the total number of bins to a manageable level.  Finally, the matrix $\textbf{R}$ only depends on the unfolded features and not on any other auxiliary features that may be useful for determining the detector response.  Even though the inputs to the unfolding may be calibrated, if the detector response depends on additional features, the result will be suboptimal and potentially biased.

OmniFold is an approach that addresses all three of the above challenges by using neural networks.  Like IBU, OmniFold is an iterative method.  Furthermore, OmniFold is a naturally generalization of IBU in the sense that if you apply the OmniFold method to binned data, it reproduces IBU at each iteration.  The OmniFold method proceeds as follows:

\begin{description}
\item Iterate:
	\begin{enumerate}[label={(\arabic*)}]
		\item Reweight detector-level simulation to match data.
		\item Use the weights from (1) to reweight the default particle-level simulation to weighted simulation.
	\end{enumerate}
\end{description}

As an algorithm, the OmniFold procedure does not require neural networks; however, these tools are well-suited to be the reweighting functions described above (more on this below).  Step (2) of the OmniFold algorithm is important because the weights from (1) are not a function of the particle-level phase space.  In particular, two events in the same particle-level phase space can be mapped to different detector-level events.  However, to be a proper unfolding, we require that two events with the same particle-level phase space have the same weight.  This is achieved with Step (2).

The OmniFold procedure is illustrated schematically in Fig.~\ref{fig:omnifoldschematic}.  The detector-level weights at a given step are represented by the symbol $\omega_n$ while the particle-level weights are represented by the symbol $\nu_n$.  Note that for Step (2), we could reweight from the original particle-level simulation to the weighted simulation or from the weights derived at the previous step to weights from (1).  Both achieve the same results, but there may be advantages to learning an incremental reweighting (as in Fig.~\ref{fig:omnifoldschematic}) since each step is a correction on the previous step.

The output of the OmniFold method is a set of events with weights.  One can then make any observables from the features used in the unfolding and from those features, one can construct histograms using whatever binning is desired.

\begin{figure}[h!]
\centering
\includegraphics[width=0.6\textwidth]{Figures/schematic.pdf}
\caption{Figure adapted from Ref.~\cite{1911.09107}.}
\label{fig:omnifoldschematic}
\end{figure}

An effective strategy for implementing the OmniFold protocol is to use neural networks as reweighting functions.  A reweighting function is simply a ratio of two probability densities.  It is a fact that the optimal classifier between two event samples is the same quantity - the likelihood ratio (or any monotonic function of it).  Therefore, we build a reweighting function by training a classifier and properly interpreting the output.  In particular, a classifier $f$ trained using the standard binary cross entropy loss function

\begin{align}
f^*=\argmin_f-\sum_{i\in\text{dataset 1}} \log(g(x_i)) -\sum_{i\in\text{dataset 2}} \log(1-f(x_i)),
\end{align}
%
has the property\footnote{This fact is well-known~\cite{hastie01statisticallearning,sugiyama_suzuki_kanamori_2012} and also has been used in many contexts in high-energy physics~\cite{2010.03569,1907.08209,Stoye:2018ovl,Hollingsworth:2020kjg,Brehmer:2018kdj,Brehmer:2018eca,Brehmer:2019xox,Brehmer:2018hga,Cranmer:2015bka,Badiali:2020wal,Andreassen:2020nkr,Andreassen:2019cjw,Fischer-ACAT2019}.} that asymptotically (enough training data, flexible enough network architecture and training):

\begin{align}
\frac{f^*(x)}{1-f^*(x)}\propto \frac{p(x|\text{dataset 1})}{p(x|\text{dataset 2})}.
\end{align}
%
The proportionality is equality if the same number of events are in the two datasets.  A constant is not important from the point of view of reweighting.

The power of neural networks is that they are naturally unbinned and can process high-dimensional data.  Furthermore, neural networks can also process variable-length data.  We will refer to the case that the inputs are one-dimensional as UniFold, the case that the inputs are multi- (but fixed-)dimensional as MultiFold, and the case where there are.  If it is clear from context, we will simply refer to all of these approaches as OmniFold.

To get a feel for what OmniFold does, the next section compares various approaches in a two-bin example where it is easy to derive the steps analytically.  Following that, we show a Gaussian example where the power of the neural network becomes clear.

\subsubsection{Illustration of Different Methods}

In order to illustrate the OmniFold procedure and how it compares to other techniques, it is useful to consider a simple two-bin example.  Suppose that there are only two possible values at particle-level and detector-level: $(T_1,T_2)$ and $(R_1,R_2)$, respectively.  Further suppose that the detector response is given by:

\begin{align}
\Pr(R_1|T_1)&=100\%\,,\\
\Pr(R_1|T_2)&=50\%\,.
\end{align}

The simulation has $\Pr_\text{MC}(T_i)=50\%$ so that $\Pr_\text{MC}(R_1)=75\%$ and $\Pr_\text{MC}(R_2)=25\%$.   Finally, we observe $\Pr_\text{Data}(R_1)=50\%$ in data.  What do the various method predict for $\Pr_\text{unfolded}(T_1)$?   Before proceeding, note that the correct answer is $\Pr_\text{Data}(T_1)=0$.

\paragraph{OmniFold}  The first step of OmniFold is to derive weights $\omega_1$ to make the MC match the data.  The weight function is specified by two numbers, one for each of the two bin values.  These weights are given by $\omega_1(R_i)=\Pr_\text{Data}(R_i)/\Pr_\text{MC}(R_i)$, which is $\omega_1(R_1)=2/3$ and $\omega_1(R_2)=2$.  These weights are then pulled back to particle level.  In MC, 50\% of events have $(T_1,R_1)$, 25\% of events have $(T_2,R_1)$ and 25\% of events have $(T_2,R_2)$.   The first two of these types of events get assigned $\omega(R_1)$ and the last one gets assigned $\omega(R_2)$.  Therefore, the weighted particle-level probability mass function is $\Pr_\text{MC,1}(T_1)=1/3$.  The second step of OmniFold derives weights $\nu_1=\Pr_\text{MC,1}(T_i)/\Pr_\text{MC}(T_i)$, which are $\nu_1(T_1)=2/3$ and $\nu_1(T_2)=4/3$.

The above procedure is then repeated using the weights $\nu_1$ pushed to detector-level.  The new detector-level probability mass is given by $\Pr_\text{MC,2}(R_1)=2/3$.  Detector-level weights are derived according to $\omega_2(R_i)=\Pr_\text{Data}(R_i)/\Pr_\text{MC,2}(R_i)$.  Table~\ref{lab:omnifoldexample} shows the evolution of OmniFold over many iterations.

\begin{table}[h!]
\centering
\begin{tabular}{|ccccccc| }
\hline
$i$ & $\omega_i(R_1)$ & $\omega_i(R_2)$ & $\nu_i(R_1)$ & $\nu_i(R_2)$ & $\Pr_\text{MC,$i$}(R_1)$ & $\Pr_\text{MC,$i$}(T_1)$ \\
\hline
0 & 1 & 1 & 1 & 1 & $\sfrac{3}{4}$ & $\sfrac{1}{2}$ \\
1 & $\sfrac{2}{3}$ & $2$ & $\sfrac{2}{3}$ & $\sfrac{4}{3}$ & $\sfrac{2}{3}$ & $\sfrac{1}{3}$ \\
2 & $\sfrac{3}{4}$ & $\sfrac{3}{2}$ & $\sfrac{1}{2}$ & $\sfrac{3}{2}$ & $\sfrac{5}{8}$ & $\sfrac{1}{4}$ \\
\vdots & \vdots & \vdots & \vdots & \vdots & \vdots & \vdots \\
$\infty$ & 1 & 1 & 0 & 2 & \sfrac{1}{2} & 0\\
\hline
\end{tabular}
\caption{The evolution of OmniFold for the simple two-bin example described in the text.}
\label{lab:omnifoldexample}
\end{table}

\paragraph{Conditional GAN Unfolding (CGU)} In the training phase of CGU, one learns $\Pr(T_i|R_j)$ based on the simulation.  In the simple binned case, this probability mass is specified by four numbers.

\begin{align}
\Pr(T_i|R_j)&=\frac{\Pr(R_j|T_i)\Pr(T_i)}{\Pr(R_j|T_1)\Pr(T_1)+\Pr(R_j|T_2)\Pr(T_2)}\\
&=\frac{\Pr(R_j|T_i)}{\Pr(R_j|T_1)+\Pr(R_j|T_2)}\\
&=\left\{\begin{matrix}\sfrac{2}{3} & \text{$i=1$ and $j=1$} \cr 0 & \text{$i=1$ and $j=2$}  \cr \sfrac{1}{3}& \text{$i=2$ and $j=1$}  \cr 1& \text{$i=2$ and $j=2$}  \end{matrix}\right.
\end{align}
%
Applied to data, we would measure
%
\begin{align}
\Pr{}_\text{unfolded}(T_1)&=\Pr(T_1|R_1)\Pr{}_\text{data}(R_1)+\Pr(T_1|R_2)\Pr{}_\text{data}(R_2)=\sfrac{1}{6}\,,\\
\Pr{}_\text{unfold}(T_2)&=\Pr(T_2|R_1)\Pr{}_\text{data}(R_1)+\Pr(T_2|R_2)\Pr{}_\text{data}(R_2)=\sfrac{5}{6}\,,
\end{align}
%
which is the wrong answer.

\paragraph{Conditional Normalizing Flow Unfolding (CNFU)} In the binned case, this is equivalent to matrix inversion.   The response matrix is

\begin{align}
\mathbf{R}=\begin{pmatrix} 1& \sfrac{1}{2}\cr 0 & \sfrac{1}{2}\end{pmatrix}\implies \mathbf{R}^{-1}=\begin{pmatrix} 1 & -1 \cr 0 & 2\end{pmatrix}\,.
\end{align}
%
Applying $\mathbf{R}^{-1}$ to $(\sfrac{1}{2},\sfrac{1}{2}$) results in $(0,1)$, the correct answer.  There are many undesirable features of matrix inversion, but it does result in an unbiased measurement.

\paragraph{Iterative Bayesian Unfolding (IBU)}

Table~\ref{lab:ibuexample} shows the evolution of IBU over many iterations.  Note that the second column of Table~\ref{lab:ibuexample} is the same as the last column of Table~\ref{lab:omnifoldexample} - this is because OmniFold and IBU are equivalent in the binned case.

\begin{table}[h!]
\centering
\begin{tabular}{|cccccc| }
\hline
$i$ & $\Pr_\text{MC,$i$}(T_1)$ & $\Pr_0(T_1|M_1)$ & $\Pr_0(T_1|M_2)$ & $\Pr_0(T_2|M_1)$ & $\Pr_0(T_2|M_2)$\\
\hline
$0$ & $\sfrac{1}{2}$ & $\sfrac{2}{3}$ & 0 & $\sfrac{1}{3}$ & $1$\\
$1$ & $\sfrac{1}{3}$ & $\sfrac{1}{2}$ & 0 & $\sfrac{1}{2}$ & $1$\\
$2$ & $\sfrac{1}{4}$ & $\sfrac{2}{5}$ & 0 & $\sfrac{3}{5}$ & $1$\\
\vdots & \vdots & \vdots & \vdots & \vdots & \vdots  \\
$\infty$ & 0 & 0 & 0 & 1 & 1\\
\hline
\end{tabular}
\caption{The evolution of IBU for the simple two-bin example described in the text.}
\label{lab:ibuexample}
\end{table}

\subsection{Gaussian Example}

This section shows the unfolding of a one-dimensional Gaussian random variable with a Gaussian noise model.  Figure~\ref{fig:gaussian:inputs} shows the `data' and `simulation'.  The data has a mean of 1 and a standard deviation of 1.5 while the simulation has a mean of 0 and a standard deviation of 1.  In both cases, the noise is an additive Gaussian with mean 0 and unit variance.

Describe the neural network.  3 layers, 50 nodes per layer, ReLU on intermediate layers, sigmoid on output layer, train for 200 epochs with early stopping with a patience of 10 (always stopped early after $\mathcal{O}(10)$ iterations).  Binary cross entropy, adam, batch size 10000, one million events total.  Keras + Tensorflow.

OmniFold is demonstrated in Fig.~\ref{fig:gaussian:iterations}.

\begin{figure}[h!]
\centering
\includegraphics[width=0.6\textwidth]{Figures/GaussianToyExample/GaussianToyExample-Distributions.pdf}
\caption{The `simulation' (left) and `data' (right) for the one-dimensional Gaussian illustration.  The narrower histograms are the particle-level distributions and the wider ones are the detector-level distributions.}
\label{fig:gaussian:inputs}
\end{figure}

\begin{figure}[h!]
\centering
\subfloat[1 iteration]{\includegraphics[width=0.45\textwidth]{Figures/GaussianToyExample/GaussianToyExample-UnfoldingResultsIteration01.pdf}}\subfloat[2 iterations]{\includegraphics[width=0.45\textwidth]{Figures/GaussianToyExample/GaussianToyExample-UnfoldingResultsIteration02.pdf}}\\
\subfloat[3 iterations]{\includegraphics[width=0.45\textwidth]{Figures/GaussianToyExample/GaussianToyExample-UnfoldingResultsIteration03.pdf}}\subfloat[4 iterations]{\includegraphics[width=0.45\textwidth]{Figures/GaussianToyExample/GaussianToyExample-UnfoldingResultsIteration04.pdf}}\\
\subfloat[5 iterations]{\includegraphics[width=0.45\textwidth]{Figures/GaussianToyExample/GaussianToyExample-UnfoldingResultsIteration05.pdf}}\subfloat[6 iterations]{\includegraphics[width=0.45\textwidth]{Figures/GaussianToyExample/GaussianToyExample-UnfoldingResultsIteration06.pdf}}
\caption{An illustration of six iterations of the OmniFold algorithm to the one-dimensional Gaussian example.  For each iteration, the left plot is the detector-level distribution with weights $\omega_n$ and the right plot is the particle-level distribution with weights $\nu_n$.}
\label{fig:gaussian:iterations}
\end{figure}

%use subfig

\subsection{OmniFold for ATLAS}

Describe the NN setup.  Also need to say how we deal with acceptance effects.

\subsection{Technical Closure Test}

\subsection{Stress Tests}

\section{Systematic uncertainties}
\label{sec:uncerts}

\subsection{Pileup Reweighting Uncertainties}
As discussed in section~\ref{subsec:PRW}, a scaling correction is applied to the data during pileup reweighting. The up and down systematic variations for this scaling are
stored as PRW\_DATASF.

\subsection{Muon Uncertainties}

\subsubsection{Muon Efficiency Uncertainties}
As summarized in section~\ref{subsec:MCCorr} there are several scale factors associated with muon reconstruction related to efficiencies in the trigger, reconstruction, isolation and track-to-vertex association.
As such, there are 2 systematics associated with the trigger scale factor, 2 for the isolation scale factor, 4 for the reconstruction efficiency scale factor, and 2 for the TTVA scale factor.
\begin{itemize}
  \item MUON\_EFF\_TrigSystUncertainty
  \item MUON\_EFF\_TrigStatUncertainty
  \item MUON\_EFF\_ISO\_SYS
  \item MUON\_EFF\_ISO\_STAT
  \item MUON\_EFF\_RECO\_SYS
  \item MUON\_EFF\_RECO\_SYS\_LOWPT
  \item MUON\_EFF\_RECO\_STAT
  \item MUON\_EFF\_RECO\_STAT\_LOWPT
  \item MUON\_EFF\_TTVA\_SYS
  \item MUON\_EFF\_TTVA\_STAT
\end{itemize}

\subsubsection{Muon Calibration Uncertainties}
There are also systematic variations associated with the calibration applied to the muons. These will have an affect on the muon kinematics. The systematic variations are listed below, and include
one for momentum variations due to measurements made in the inner detector, one for momentum effects related to the muon spectrometer, 2 systematics related to the measured sagitta value, and one for the
scale.
\begin{itemize}
  \item MUON\_ID
  \item MUON\_MS
  \item MUON\_SAGITTA\_RESBIAS
  \item MUON\_SAGITTA\_RHO
  \item MUON\_SCALE
\end{itemize}

\subsection{Track Uncertainties}
The first five systematics listed below are related to the efficiency and fake rate for the tracks. The final systematic is related to alignment corrections made to the tracks, and biases the $\pt$ of the track. 
\begin{itemize}
  \item InDet::TRK\_EFF\_TIGHT\_GLOBAL
  \item InDet::TRK\_EFF\_TIGHT\_IBL
  \item InDet::TRK\_EFF\_TIGHT\_PP0
  \item InDet::TRK\_EFF\_TIGHT\_PHYSMODEL
  \item InDet::TRK\_EFF\_LOOSE\_TIDE
  \item InDet::TRK\_FAKE\_RATE\_LOOSE
  \item InDet::TRK\_BIAS\_QOVERP\_SAGITTA\_WM
\end{itemize}

\subsection{Modeling and Unfolding Uncertainties}

%-------------------------------------------------------------------------------
\section{Results}
\label{sec:result}
%-------------------------------------------------------------------------------

%-------------------------------------------------------------------------------
\section{Conclusion}
\label{sec:conclusion}
%-------------------------------------------------------------------------------

Place your conclusion here.


%-------------------------------------------------------------------------------
% If you use biblatex and either biber or bibtex to process the bibliography
% just say \printbibliography here
\printbibliography
% If you want to use the traditional BibTeX you need to use the syntax below.
%\bibliographystyle{bib/bst/atlasBibStyleWithTitle}
%\bibliography{ANA-STDM-2020-17-INT1,bib/ATLAS,bib/CMS,bib/ConfNotes,bib/PubNotes}
%-------------------------------------------------------------------------------

%-------------------------------------------------------------------------------
% Print the list of contributors to the analysis
% The argument gives the fraction of the text width used for the names
%-------------------------------------------------------------------------------
\clearpage
%The supporting notes for the analysis should also contain a list of contributors.
%This information should usually be included in \texttt{mydocument-metadata.tex}.
%The list should be printed either here or before the Table of Contents.
%\PrintAtlasContribute{0.30}


%-------------------------------------------------------------------------------
\clearpage
\appendix
\part*{Appendices}
\addcontentsline{toc}{part}{Appendices}
%-------------------------------------------------------------------------------
\section{Particle composition and detector response of hadronic recoil}

This section examines the particle composition and the detector response of the leading jet that balance the $Z$~boson in the fiducial region of the analysis.
The Powheg $Z\to \mu\mu$ sample is used. The particle origin of each track is obtained using the \texttt{truthParticleLink} to find the truth identification of each reconstructed track (see section~\ref{sec:samples}). In addition to the standard event preselection described in section~\ref{sec:selection}, extra cuts are used to make sure the leading reconstructed jet matches the leading particle jet in each event.
%To ensure the closet matching between the leading reconstructed jet and leading truth jet, and always have very good DR between them, the cut on ratio of
This is achieved by requiring $|y_\mathrm{j1}|<2.1$ and $\pTlj / \pTsj < 0.7$. The selection is applied on reconstructed level. However, due to the additional criterion, the leading particle-level (truth) jet will always match the leading reco jet. Note that the jets will have a very high \pt{} since the event selection includes the selection $\pTll > \SI{165}{\GeV}$, and the jet will have a similar \pt{} to that of the $Z$~boson.

Particles are classified depending to their type in the following categories: \texttt{pion} for $\pi^+$ and $\pi^-$; \texttt{kaon} for charged kaons; \texttt{proton};
\texttt{mu} for $\mu^-$ or $\mu^+$; \texttt{e} for $e^-$ or $e^+$;
strange for any charge hadron with strange content other than kaons; \texttt{neutral} for any neutral hadron; \texttt{gam} for photons. As mentioned above, tracks are classified in the same categories using the \texttt{truthParticleLink}. In case there is no truth particle match, it is labelled unmatched. This includes contributions from pileup particles and fake tracks (and possibly secondaries).


All of the reconstructed tracks inside leading jet as well as charged-particles in the associated particle-level jets that are not matched to reconstructed tracks are used for the plots shown in this section. Figure~\ref{fig:fraction of charged particles in reco jet} shows the fraction of the particle multiplicity (left) as a function of the truth jet \pT. The 78\% of reconstructed tracks make up pions about of the total, kaons are about 12\% of the total, protons are about 4\%, unmatched tracks are around 5\% and strange are almost 1\%. The momentum fraction as a function of jet pT is shown in right in Figure~\ref{fig:fraction of charged particles in reco jet}. The momentum fraction for different particles varies slightly as compare to multiplicity fraction.

\begin{figure}[b]
	\centering
	\includegraphics[width=0.48\textwidth,page=3]{figures/jet_comp_study_powheg_Tight_MultiplicityFraction.pdf}
	\includegraphics[width=0.48\textwidth,page=3]{figures/jet_comp_study_powheg_Tight_pTFraction.pdf}
	\caption{Composition of the leading jet in terms of particle multiplicity (left) and \pt{} (right) as a function of jet \pt{}.
		The photon multiplicity is divided by 2 in the left plot, since most photons are produced in pairs by a neutral hadron (e.g.\ $\pi\to\gamma\gamma$).
		We can see that $\approx 42$\% of the jet \pt{} is carried by neutral particles. It's surprising that the fraction of neutral hadrons is so high (why??).}
	\label{fig:truthJetComp}
\end{figure}


\begin{figure}
\centering
\includegraphics[width=0.48\textwidth,page=1]{figures/jet_comp_study_powheg_Tight_MultiplicityFraction.pdf}
\includegraphics[width=0.48\textwidth,page=1]{figures/jet_comp_study_powheg_Tight_pTFraction.pdf}
\caption {The fraction of the reconstructed tracks multiplicity (left) and \pT (right) inside reconstructed leading jet as a function of jet pT for various categories (see the text for details).}
\label{fig:fraction of charged particles in reco jet}
\end{figure}


\begin{figure}
\centering
\includegraphics[scale=0.3, page=2]{figures/jet_comp_study_powheg_Tight_MultiplicityFraction.pdf}
\hspace{2mm}
\includegraphics[scale=0.3, page=2]{figures/jet_comp_study_powheg_Tight_pTFraction.pdf}
\caption {The fraction of the charged particle multiplicity (left) and \pT (right) inside particle level leading jet as a function of jet pT for various categories (see the text for details).}
\label{fig:fraction of charged particles in truth jet}
\end{figure}

Figure~\ref{fig:fraction of charged particles in truth jet} shows the fraction of the charged particle multiplicity (left) and their \pT  as a function of the truth jet \pT in the particle-level jet. From charged particle compositions, mostly are pions ~74\%, kaons are 14\% and rest 12\% comprise other charged particle.
Figure~\ref{fig:fraction of charged particles in truth jet} shows similar plots for particle level jet, but for all the particle compositions in the leading jet. From the multiplicity fraction plot (left), it can be seen that ~43\% are photons which are ~20\% greater than in \pt fraction (right) dur to decay of pions into photons.


Figures~\ref{fig:response pion and kaon} to ~\ref{fig:track jet response} are showing the detector response for each particle category. The pion and kaon respose are ~93\% and ~94\% respectively which indicates tracking efficiency is around 93\%. Figure~\ref{fig:unmatched tracks fraction} depicts fraction of unmatched track w.r.t reco level jet, which indicates ~2\% tracking inefficiency. Similarly, figures~\ref{fig:fraction pions and kaons} to ~\ref{fig:response electrons and starnge particles} are showing the fraction of individual particles at truth level and reconstructed level. As shown in figure~\ref{fig:fraction of charged particles in truth jet}, pions make most of jet fraction, and for muons, electrons and strange particles this fraction is much lower. the muon and electron responses are very low because these particles come from semileptonic decay of heavy hadrons, so we have a displaced vertex

\begin{figure}
\centering
\includegraphics[scale=0.3, page=1]{figures/jet_comp_study_powheg_Tight_MultiplicityFraction_withLooseandTight.pdf}
\includegraphics[scale=0.3, page=2]{figures/jet_comp_study_powheg_Tight_MultiplicityFraction_withLooseandTight.pdf}
\caption {The response of pions (left) and  kaons (right) as a function of jet \pT.}
\label{fig:response pion and kaon}
\end{figure}

\begin{figure}
\centering
\includegraphics[scale=0.3, page=3]{figures/jet_comp_study_powheg_Tight_MultiplicityFraction_withLooseandTight.pdf}
\includegraphics[scale=0.3, page=4]{figures/jet_comp_study_powheg_Tight_MultiplicityFraction_withLooseandTight.pdf}
\caption {The response of protons (left) and  muons (right) as a function of jet \pT.}
\label{fig:response proton and muon}
\end{figure}

\begin{figure}
\centering
\includegraphics[scale=0.3, page=5]{figures/jet_comp_study_powheg_Tight_MultiplicityFraction_withLooseandTight.pdf}
\includegraphics[scale=0.3, page=6]{figures/jet_comp_study_powheg_Tight_MultiplicityFraction_withLooseandTight.pdf}
\caption {The response of electrons (left) and strange particles (right) as a function of jet \pT.}
\label{fig:response electron and starnge}
\end{figure}

\begin{figure}
\centering
\includegraphics[scale=0.3, page=7]{figures/jet_comp_study_powheg_Tight_MultiplicityFraction_withLooseandTight.pdf}
\caption {The response of  all the reconstructed tracks w.r.t charged particles as a function of jet \pT.}
\label{fig:track jet response}
\end{figure}

\begin{figure}
\centering
\includegraphics[scale=0.3, page=8]{figures/jet_comp_study_powheg_Tight_MultiplicityFraction_withLooseandTight.pdf}
\caption {The fraction of unmatched tracks w.r.t reco level jet as a function of jet \pT.}
\label{fig:unmatched tracks fraction}
\end{figure}


\begin{figure}
\centering
\includegraphics[scale=0.3, page=4]{figures/jet_comp_study_powheg_Tight_pTFraction.pdf}
\includegraphics[scale=0.3, page=5]{figures/jet_comp_study_powheg_Tight_pTFraction.pdf}
\caption {The fraction of pions (left) and kaons (right) in the reconstructed tracks as a function of jet \pT.}
\label{fig:fraction pions and kaons}
\end{figure}


\begin{figure}
\centering
\includegraphics[scale=0.3, page=6]{figures/jet_comp_study_powheg_Tight_pTFraction.pdf}
\includegraphics[scale=0.3, page=7]{figures/jet_comp_study_powheg_Tight_pTFraction.pdf}
\caption {The fraction of protons (left) and muons (right) in the reconstructed tracks as a function of jet \pT.}
\label{fig:response protons and muons}
\end{figure}


\begin{figure}
\centering
\includegraphics[scale=0.3, page=8]{figures/jet_comp_study_powheg_Tight_pTFraction.pdf}
\includegraphics[scale=0.3, page=9]{figures/jet_comp_study_powheg_Tight_pTFraction.pdf}
\caption {The fraction of electrons (left) and strange particles (right) in the reconstructed tracks as a function of jet \pT.}
\label{fig:response electrons and starnge particles}
\end{figure}

\clearpage

\section{Data and Monte Carlo samples}
\label{app:datasets}

\subsection{Data}
\begin{tiny}
\begin{verbatim}
data15_13TeV.periodD.physics_Main.PhysCont.DAOD_STDM7.grp23_v01_4397
data15_13TeV.periodE.physics_Main.PhysCont.DAOD_STDM7.grp23_v01_4397
data15_13TeV.periodF.physics_Main.PhysCont.DAOD_STDM7.grp23_v01_4397
data15_13TeV.periodG.physics_Main.PhysCont.DAOD_STDM7.grp23_v01_4397
data15_13TeV.periodH.physics_Main.PhysCont.DAOD_STDM7.grp23_v01_4397
data15_13TeV.periodJ.physics_Main.PhysCont.DAOD_STDM7.grp23_v01_4397

data16_13TeV.periodA.physics_Main.PhysCont.DAOD_STDM7.grp23_v01_4397
data16_13TeV.periodB.physics_Main.PhysCont.DAOD_STDM7.grp23_v01_4397
data16_13TeV.periodC.physics_Main.PhysCont.DAOD_STDM7.grp23_v01_4397
data16_13TeV.periodD.physics_Main.PhysCont.DAOD_STDM7.grp23_v01_4397
data16_13TeV.periodE.physics_Main.PhysCont.DAOD_STDM7.grp23_v01_4397
data16_13TeV.periodF.physics_Main.PhysCont.DAOD_STDM7.grp23_v01_4397
data16_13TeV.periodG.physics_Main.PhysCont.DAOD_STDM7.grp23_v01_4397
data16_13TeV.periodI.physics_Main.PhysCont.DAOD_STDM7.grp23_v01_4397
data16_13TeV.periodK.physics_Main.PhysCont.DAOD_STDM7.grp23_v01_4397
data16_13TeV.periodL.physics_Main.PhysCont.DAOD_STDM7.grp23_v01_4397

data17_13TeV.periodB.physics_Main.PhysCont.DAOD_STDM7.grp23_v01_p4397
data17_13TeV.periodC.physics_Main.PhysCont.DAOD_STDM7.grp23_v01_p4397
data17_13TeV.periodD.physics_Main.PhysCont.DAOD_STDM7.grp23_v01_p4397
data17_13TeV.periodE.physics_Main.PhysCont.DAOD_STDM7.grp23_v01_p4397
data17_13TeV.periodF.physics_Main.PhysCont.DAOD_STDM7.grp23_v01_p4397
data17_13TeV.periodH.physics_Main.PhysCont.DAOD_STDM7.grp23_v01_p4397
data17_13TeV.periodI.physics_Main.PhysCont.DAOD_STDM7.grp23_v01_p4397
data17_13TeV.periodK.physics_Main.PhysCont.DAOD_STDM7.grp23_v01_p4397

data18_13TeV.periodB.physics_Main.PhysCont.DAOD_STDM7.grp23_v01_p4397
data18_13TeV.periodC.physics_Main.PhysCont.DAOD_STDM7.grp23_v01_p4397
data18_13TeV.periodD.physics_Main.PhysCont.DAOD_STDM7.grp23_v01_p4397
data18_13TeV.periodF.physics_Main.PhysCont.DAOD_STDM7.grp23_v01_p4397
data18_13TeV.periodI.physics_Main.PhysCont.DAOD_STDM7.grp23_v01_p4397
data18_13TeV.periodK.physics_Main.PhysCont.DAOD_STDM7.grp23_v01_p4397
data18_13TeV.periodL.physics_Main.PhysCont.DAOD_STDM7.grp23_v01_p4397
data18_13TeV.periodM.physics_Main.PhysCont.DAOD_STDM7.grp23_v01_p4397
data18_13TeV.periodO.physics_Main.PhysCont.DAOD_STDM7.grp23_v01_p4397
data18_13TeV.periodQ.physics_Main.PhysCont.DAOD_STDM7.grp23_v01_p4397

\end{verbatim}
\end{tiny}

\subsection{Monte Carlo}

\subsubsection{Powheg+Pythia8 samples}

\begin{tiny}
\begin{verbatim}
mc16_13TeV.361107.PowhegPythia8EvtGen_AZNLOCTEQ6L1_Zmumu.deriv.DAOD_STDM7.e3601_s3126_r9364_p4145
mc16_13TeV.361107.PowhegPythia8EvtGen_AZNLOCTEQ6L1_Zmumu.deriv.DAOD_STDM7.e3601_s3126_r10201_p4145
mc16_13TeV.361107.PowhegPythia8EvtGen_AZNLOCTEQ6L1_Zmumu.deriv.DAOD_STDM7.e3601_s3126_r10724_p4145
\end{verbatim}
\end{tiny}

\subsubsection{Sherpa 2.2.1 samples}

\begin{tiny}
\begin{verbatim}
mc16_13TeV.364100.Sherpa_221_NNPDF30NNLO_Zmumu_MAXHTPTV0_70_CVetoBVeto.deriv.DAOD_STDM7.e5271_s3126_r9364_p4357
mc16_13TeV.364101.Sherpa_221_NNPDF30NNLO_Zmumu_MAXHTPTV0_70_CFilterBVeto.deriv.DAOD_STDM7.e5271_s3126_r9364_p4357
mc16_13TeV.364102.Sherpa_221_NNPDF30NNLO_Zmumu_MAXHTPTV0_70_BFilter.deriv.DAOD_STDM7.e5271_s3126_r9364_p4357
mc16_13TeV.364103.Sherpa_221_NNPDF30NNLO_Zmumu_MAXHTPTV70_140_CVetoBVeto.deriv.DAOD_STDM7.e5271_s3126_r9364_p4357
mc16_13TeV.364104.Sherpa_221_NNPDF30NNLO_Zmumu_MAXHTPTV70_140_CFilterBVeto.deriv.DAOD_STDM7.e5271_s3126_r9364_p4357
mc16_13TeV.364105.Sherpa_221_NNPDF30NNLO_Zmumu_MAXHTPTV70_140_BFilter.deriv.DAOD_STDM7.e5271_s3126_r9364_p4357
mc16_13TeV.364106.Sherpa_221_NNPDF30NNLO_Zmumu_MAXHTPTV140_280_CVetoBVeto.deriv.DAOD_STDM7.e5271_s3126_r9364_p4357
mc16_13TeV.364107.Sherpa_221_NNPDF30NNLO_Zmumu_MAXHTPTV140_280_CFilterBVeto.deriv.DAOD_STDM7.e5271_s3126_r9364_p4357
mc16_13TeV.364108.Sherpa_221_NNPDF30NNLO_Zmumu_MAXHTPTV140_280_BFilter.deriv.DAOD_STDM7.e5271_s3126_r9364_p4357
mc16_13TeV.364109.Sherpa_221_NNPDF30NNLO_Zmumu_MAXHTPTV280_500_CVetoBVeto.deriv.DAOD_STDM7.e5271_s3126_r9364_p4357
mc16_13TeV.364110.Sherpa_221_NNPDF30NNLO_Zmumu_MAXHTPTV280_500_CFilterBVeto.deriv.DAOD_STDM7.e5271_s3126_r9364_p4357
mc16_13TeV.364111.Sherpa_221_NNPDF30NNLO_Zmumu_MAXHTPTV280_500_BFilter.deriv.DAOD_STDM7.e5271_s3126_r9364_p4357
mc16_13TeV.364112.Sherpa_221_NNPDF30NNLO_Zmumu_MAXHTPTV500_1000.deriv.DAOD_STDM7.e5271_s3126_r9364_p4357
mc16_13TeV.364113.Sherpa_221_NNPDF30NNLO_Zmumu_MAXHTPTV1000_E_CMS.deriv.DAOD_STDM7.e5271_s3126_r9364_p4357

mc16_13TeV.364100.Sherpa_221_NNPDF30NNLO_Zmumu_MAXHTPTV0_70_CVetoBVeto.deriv.DAOD_STDM7.e5271_s3126_r10201_p4357
mc16_13TeV.364101.Sherpa_221_NNPDF30NNLO_Zmumu_MAXHTPTV0_70_CFilterBVeto.deriv.DAOD_STDM7.e5271_s3126_r10201_p4357
mc16_13TeV.364102.Sherpa_221_NNPDF30NNLO_Zmumu_MAXHTPTV0_70_BFilter.deriv.DAOD_STDM7.e5271_s3126_r10201_p4357
mc16_13TeV.364103.Sherpa_221_NNPDF30NNLO_Zmumu_MAXHTPTV70_140_CVetoBVeto.deriv.DAOD_STDM7.e5271_s3126_r10201_p4357
mc16_13TeV.364104.Sherpa_221_NNPDF30NNLO_Zmumu_MAXHTPTV70_140_CFilterBVeto.deriv.DAOD_STDM7.e5271_s3126_r10201_p4357
mc16_13TeV.364105.Sherpa_221_NNPDF30NNLO_Zmumu_MAXHTPTV70_140_BFilter.deriv.DAOD_STDM7.e5271_s3126_r10201_p4357
mc16_13TeV.364106.Sherpa_221_NNPDF30NNLO_Zmumu_MAXHTPTV140_280_CVetoBVeto.deriv.DAOD_STDM7.e5271_s3126_r10201_p4357
mc16_13TeV.364107.Sherpa_221_NNPDF30NNLO_Zmumu_MAXHTPTV140_280_CFilterBVeto.deriv.DAOD_STDM7.e5271_s3126_r10201_p4357
mc16_13TeV.364108.Sherpa_221_NNPDF30NNLO_Zmumu_MAXHTPTV140_280_BFilter.deriv.DAOD_STDM7.e5271_s3126_r10201_p4357
mc16_13TeV.364109.Sherpa_221_NNPDF30NNLO_Zmumu_MAXHTPTV280_500_CVetoBVeto.deriv.DAOD_STDM7.e5271_s3126_r10201_p4357
mc16_13TeV.364110.Sherpa_221_NNPDF30NNLO_Zmumu_MAXHTPTV280_500_CFilterBVeto.deriv.DAOD_STDM7.e5271_s3126_r10201_p4357
mc16_13TeV.364111.Sherpa_221_NNPDF30NNLO_Zmumu_MAXHTPTV280_500_BFilter.deriv.DAOD_STDM7.e5271_s3126_r10201_p4357
mc16_13TeV.364112.Sherpa_221_NNPDF30NNLO_Zmumu_MAXHTPTV500_1000.deriv.DAOD_STDM7.e5271_s3126_r10201_p4357
mc16_13TeV.364113.Sherpa_221_NNPDF30NNLO_Zmumu_MAXHTPTV1000_E_CMS.deriv.DAOD_STDM7.e5271_s3126_r10201_p4357

mc16_13TeV.364100.Sherpa_221_NNPDF30NNLO_Zmumu_MAXHTPTV0_70_CVetoBVeto.deriv.DAOD_STDM7.e5271_s3126_r10724_p4357
mc16_13TeV.364101.Sherpa_221_NNPDF30NNLO_Zmumu_MAXHTPTV0_70_CFilterBVeto.deriv.DAOD_STDM7.e5271_s3126_r10724_p4357
mc16_13TeV.364102.Sherpa_221_NNPDF30NNLO_Zmumu_MAXHTPTV0_70_BFilter.deriv.DAOD_STDM7.e5271_s3126_r10724_p4357
mc16_13TeV.364103.Sherpa_221_NNPDF30NNLO_Zmumu_MAXHTPTV70_140_CVetoBVeto.deriv.DAOD_STDM7.e5271_s3126_r10724_p4357
mc16_13TeV.364104.Sherpa_221_NNPDF30NNLO_Zmumu_MAXHTPTV70_140_CFilterBVeto.deriv.DAOD_STDM7.e5271_s3126_r10724_p4357
mc16_13TeV.364105.Sherpa_221_NNPDF30NNLO_Zmumu_MAXHTPTV70_140_BFilter.deriv.DAOD_STDM7.e5271_s3126_r10724_p4357
mc16_13TeV.364106.Sherpa_221_NNPDF30NNLO_Zmumu_MAXHTPTV140_280_CVetoBVeto.deriv.DAOD_STDM7.e5271_s3126_r10724_p4357
mc16_13TeV.364107.Sherpa_221_NNPDF30NNLO_Zmumu_MAXHTPTV140_280_CFilterBVeto.deriv.DAOD_STDM7.e5271_s3126_r10724_p4357
mc16_13TeV.364108.Sherpa_221_NNPDF30NNLO_Zmumu_MAXHTPTV140_280_BFilter.deriv.DAOD_STDM7.e5271_s3126_r10724_p4357
mc16_13TeV.364109.Sherpa_221_NNPDF30NNLO_Zmumu_MAXHTPTV280_500_CVetoBVeto.deriv.DAOD_STDM7.e5271_s3126_r10724_p4357
mc16_13TeV.364110.Sherpa_221_NNPDF30NNLO_Zmumu_MAXHTPTV280_500_CFilterBVeto.deriv.DAOD_STDM7.e5271_s3126_r10724_p4357
mc16_13TeV.364111.Sherpa_221_NNPDF30NNLO_Zmumu_MAXHTPTV280_500_BFilter.deriv.DAOD_STDM7.e5271_s3126_r10724_p4357
mc16_13TeV.364112.Sherpa_221_NNPDF30NNLO_Zmumu_MAXHTPTV500_1000.deriv.DAOD_STDM7.e5271_s3126_r10724_p4357
mc16_13TeV.364113.Sherpa_221_NNPDF30NNLO_Zmumu_MAXHTPTV1000_E_CMS.deriv.DAOD_STDM7.e5271_s3126_r10724_p4357
\end{verbatim}
\end{tiny}

\subsubsection{Powheg+Pythia8 samples for systematic uncertainties on Parton Shower}

\begin{tiny}
\begin{verbatim}
mc15_13TeV.600818.PhPy8_Zmumu_Pt100.deriv.DAOD_TRUTH1.e8357_p4161
mc15_13TeV.600817.PhPy8_Zmumu_Var2Up_Pt100.deriv.DAOD_TRUTH1.e8357_p4161
mc15_13TeV.600816.PhPy8_Zmumu_Var2Down_Pt100.deriv.DAOD_TRUTH1.e8357_p4161
mc15_13TeV.600815.PhPy8_Zmumu_Var1Down_Pt100.deriv.DAOD_TRUTH1.e8357_p4161
mc15_13TeV.600814.PhPy8_Zmumu_Var1Up_Pt100.deriv.DAOD_TRUTH1.e8357_p4161
mc15_13TeV.600813.PhPy8_Zmumu_RenDown_Pt100.deriv.DAOD_TRUTH1.e8357_p4161
mc15_13TeV.600812.PhPy8_Zmumu_RenUp_Pt100.deriv.DAOD_TRUTH1.e8357_p4161
mc15_13TeV.600810.PhPy8_Zmumu_MPIDown_Pt100.deriv.DAOD_TRUTH1.e8357_p4161
mc15_13TeV.600811.PhPy8_Zmumu_MPIUp_Pt100.deriv.DAOD_TRUTH1.e8357_p4161

\end{verbatim}
\end{tiny}

\subsubsection{Samples for Parton Distribution Function (PDF) variations}

\begin{tiny}
\begin{verbatim}
mc16_13TeV.364114.Sherpa_221_NNPDF30NNLO_Zee_MAXHTPTV0_70_CVetoBVeto.deriv.DAOD_STDM7.e5299_s3126_r10724_p4398 
mc16_13TeV.364115.Sherpa_221_NNPDF30NNLO_Zee_MAXHTPTV0_70_CFilterBVeto.deriv.DAOD_STDM7.e5299_s3126_r10724_p4398
mc16_13TeV.364116.Sherpa_221_NNPDF30NNLO_Zee_MAXHTPTV0_70_BFilter.deriv.DAOD_STDM7.e5299_s3126_r10724_p4398
mc16_13TeV.364117.Sherpa_221_NNPDF30NNLO_Zee_MAXHTPTV70_140_CVetoBVeto.deriv.DAOD_STDM7.e5299_s3126_r10724_p4398 
mc16_13TeV.364118.Sherpa_221_NNPDF30NNLO_Zee_MAXHTPTV70_140_CFilterBVeto.deriv.DAOD_STDM7.e5299_s3126_r10724_p4398
mc16_13TeV.364119.Sherpa_221_NNPDF30NNLO_Zee_MAXHTPTV70_140_BFilter.deriv.DAOD_STDM7.e5299_s3126_r10724_p4398
mc16_13TeV.364120.Sherpa_221_NNPDF30NNLO_Zee_MAXHTPTV140_280_CVetoBVeto.deriv.DAOD_STDM7.e5299_s3126_r10724_p4398
mc16_13TeV.364121.Sherpa_221_NNPDF30NNLO_Zee_MAXHTPTV140_280_CFilterBVeto.deriv.DAOD_STDM7.e5299_s3126_r10724_p4398
mc16_13TeV.364122.Sherpa_221_NNPDF30NNLO_Zee_MAXHTPTV140_280_BFilter.deriv.DAOD_STDM7.e5299_s3126_r10724_p4398
mc16_13TeV.364123.Sherpa_221_NNPDF30NNLO_Zee_MAXHTPTV280_500_CVetoBVeto.deriv.DAOD_STDM7.e5299_s3126_r10724_p4398 
mc16_13TeV.364124.Sherpa_221_NNPDF30NNLO_Zee_MAXHTPTV280_500_CFilterBVeto.deriv.DAOD_STDM7.e5299_s3126_r10724_p4398
mc16_13TeV.364125.Sherpa_221_NNPDF30NNLO_Zee_MAXHTPTV280_500_BFilter.deriv.DAOD_STDM7.e5299_s3126_r10724_p4398 
mc16_13TeV.364126.Sherpa_221_NNPDF30NNLO_Zee_MAXHTPTV500_1000.deriv.DAOD_STDM7.e5299_s3126_r10724_p4398
mc16_13TeV.364113.Sherpa_221_NNPDF30NNLO_Zee_MAXHTPTV1000_E_CMS.deriv.DAOD_STDM7.e5271_s3126_r9364_p4357
\end{verbatim}
\end{tiny}


\end{document}
