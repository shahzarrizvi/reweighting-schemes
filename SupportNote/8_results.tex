\subsection{Blinding strategy}

We have investigated data/MC differences at detector-level, but we remain blinded to the unfolded data until following the Editorial Board's approval of the procedure.

\subsection{Proposed presentation of results}

Unlike all other ATLAS measurements to date, the results here are \textbf{unbinned}.
Starting from the unbinned results, it is trivial to create histograms with any binning, which is convenient for presentation and comparison with binned measurments performed using traditional techniques.
The proposed main results to be included in the first paper is:

%Clearly, we can represent the data as histograms, but that is restrictive.

\begin{itemize}
\item
  Unbinned simultaneous unfolding of a set of 24 observables using the MultiFold technique.
  \todo{THE 24 OBSERVABLES ARE ... MOTIVATION}
\item
   A plot will be produced for each observable comparing the unfolded spectrum with a measurement performed using IBU (i.e.\ the standard technique used in ATLAS SM measurements).
   IBU requires a predefined binning, which has been derived for each observable as described in appendix~\ref{app:IBU}. This binning is presented in table~\ref{tab:IBUBins}.
%Our paper will contain a small number of unfolded histograms to illustrate the breadth of the result.
%We will show plots that show the spectra of the multifold observables, using IBU, MultiFold, and OmniFold in the support note (and also include the correlation dimension~\cite{Komiske:2019fks}) and then we will discuss with the Editorial Board which plots to go in the paper.
\item
  \todo{This is only for OmniFold, right? We'd need to make a flat n-tuple with 24 branches, I think, which should be stated here}
  We make public the weight function(s) (as a neural network, saving the model architecture and weights from TensorFlow) and the exact configuration used to generate the nominal MC events.  There will be one function per systematic uncertainty.  This, combined with a Rivet routine of the analysis selection, will ensure that the data can be used and readily and widely reinterpreted.
\item (Alternative: we make public our simulated MC sample with a vector of weights for each event)
\end{itemize}


For a second, future paper, the idea is to include...
