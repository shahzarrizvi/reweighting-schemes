\subsection{Blinding strategy}

We have investigated data/MC differences at detector-level, but we remain blinded to the unfolded data until following the Editorial Board's approval of the procedure.

\subsection{Proposed presentation of results}
\label{sec:pres}

Unlike all other ATLAS measurements to date, the results here are \textbf{unbinned}.
Starting from the unbinned results, it is trivial to create histograms with any binning, which is convenient for presentation and comparison with binned measurments performed using traditional techniques.
The proposed main results to be included in the first paper is:

%Clearly, we can represent the data as histograms, but that is restrictive.

\begin{itemize}
\item
  Unbinned simultaneous unfolding of a set of 24 observables using the MultiFold technique.
  The observables are:
  \begin{itemize}
    \item
      the transverse momentum \ptll{} and rapidity $\yll$ of the dimuon system, which probes the $Z$-boson kinematics;
    \item
      the transverse momentum (\ptlm, \ptsm),
      %OR ($p_{\mathrm{T},\mu 1}$, $p_{\mathrm{T},\mu 2}$)
      rapidity (\ylm{}, \ysm) and azimuthal angle (\philm, \phism) of the two muons probing the $Z$-boson decay kinematics;
    \item
      the transverse momentum (\ptlj, \ptsj), rapidity (\ylj{}, \ysj), azimuth (\philj, \phisj) of the two highest-\pt{} charged hadron jets in the event, which probes the jet kinematics; and
    \item
      the mass (\mlj, \msj), charged hadron multiplicity (\Nclj, \Ncsj) and $N$-subjettiness quantities \taulj{1}, %$\tau_1(j1)$, $\tau_{1,\mathrm{j1}}$
      \tausj{1}, \taulj{2}, \tausj{2}, \taulj{3} and \tausj{3} that measures the substructure of the leading two charged-hadron jets.
  \end{itemize}
\item
   A plot will be produced for each observable comparing the unfolded spectrum with a measurement performed using IBU (i.e.\ the standard technique used in ATLAS SM measurements).
   IBU requires a predefined binning, which has been derived for each observable as described in appendix~\ref{app:IBU} to achieve reasonable purity of each bin.
   This binning is presented in table~\ref{tab:IBUBins}.

   \begin{table}[h!]
       \centering
       \begin{tabular}{l|l}
       \hline\hline
       \textbf{Variable} & \textbf{Bin Edges} \\ \hline\hline
       $\pTll$ [GeV]                           & 190~ 230~ 300~ 450~ 600~ 1000 \\ \hline
       $\yll$                                  & -2.5~ -2~ -1.5~ -1~ -0.75~ -0.5~ -0.25~ 0~ 0.25~ 0.5~ 0.75~ 1~ 1.5~ 2~ 2.5 \\ \hline
       $\pTlm$ [GeV]                           & 25~ 125~ 200~ 300~ 400~ 600~ 800 \\ \hline
       $\pTsm$ [GeV]                           & 25~ 50~ 75~ 100~ 150~ 200~ 300~ 400 \\ \hline
       $\eta_{\mu1}$                           & -2.5~ -2~ -1.5~ -1~ -0.75~ -0.5~ -0.25~ 0~ 0.25~ 0.5~ 0.75~ 1~ 1.5~ 2~ 2.5 \\ \hline
       $\eta_{\mu2}$                           & -2.5~ -2~ -1.5~ -1~ -0.75~ -0.5~ -0.25~ 0~ 0.25~ 0.5~ 0.75~ 1~ 1.5~ 2~ 2.5 \\ \hline
       $\phi_{\mu1}$                           & -3.2~ -2.8~ -2.4~ -2.0~ -1.6~ -1.2~ -0.8~ -0.4~ 0~ 0.4~ 0.8~ 1.2~ 1.6~ 2.0~ \\
                                               & 2.4~ 2.8~ 3.2 \\ \hline
       $\phi_{\mu2}$                           & -3.2~ -2.8~ -2.4~ -2.0~ -1.6~ -1.2~ -0.8~ -0.4~ 0~ 0.4~ 0.8~ 1.2~ 1.6~ 2.0~ \\
                                               & 2.4~ 2.8~ 3.2 \\ \hline
       Leading track jet \pt [GeV]             & 0~ 50~ 100~ 150~ 200~ 300~ 1000 \\ \hline
       Subleading track jet \pt [GeV]          & 0~ 25~ 50~ 100~ 500 \\ \hline
       Leading track jet $y$                   & -2.5~ -2.25~ -2.0~ -1.75~ -1.5~ -1.25~ -1.0~ -0.75~ -0.5~ -0.25~ 0~ \\
                                               & 0.25~ 0.5~ 0.75~ 1.0~ 1.25~ 1.5~ 1.75~ 2.0~ 2.25~ 2.5 \\ \hline
       Subleading track jet $y$                & -2.5~ -2~ -1.5~ -1~ -0.75~ -0.5~ -0.25~ 0~ 0.25~ 0.5~ 0.75~ 1~ 1.5~ 2~ 2.5 \\ \hline
       Leading track jet $\phi$                & -3.2~ -2.8~ -2.4~ -2.0~ -1.6~ -1.2~ -0.8~ -0.4~ 0~ 0.4~ 0.8~ 1.2~ 1.6~ 2.0~ \\
                                               & 2.4~ 2.8~ 3.2 \\ \hline
       Subleading track jet $\phi$             & -3.2~ -2.8~ -2.4~ -2.0~ -1.6~ -1.2~ -0.8~ -0.4~ 0~ 0.4~ 0.8~ 1.2~ 1.6~ 2.0~ \\
                                               & 2.4~ 2.6~ 2.8~ 3.0~ 3.2 \\ \hline
       Leading track jet $n_{\text{ch}}$       & 1~ 7~ 11~ 15~ 20~ 30~ 40 \\ \hline
       Subleading track jet $n_{\text{ch}}$    & 1~ 4~ 7~ 11~ 15~ 20~ 30~ 55 \\ \hline
       Leading track jet $m$ [GeV]             & 0~ 8~ 16~ 24~ 32~ 42~ 70 \\ \hline
       Subleading track jet $m$ [GeV]          & 0~ 2.5~ 7~ 12~ 20~ 40 \\ \hline
       Leading track jet $\tau_1$              & 0~ 0.05~ 0.1~ 0.17~ 0.25~ 0.35~ 0.5~ 0.9 \\ \hline
       Subleading track jet $\tau_1$           & 0~ 0.1~ 0.2~ 0.35~ 0.5~ 0.9 \\ \hline
       Leading track jet $\tau_2$              & 0~ 0.025~ 0.05~ 0.08~ 0.12~ 0.17~ 0.25~ 0.5 \\ \hline
       Subleading track jet $\tau_2$           & 0~ 0.025~ 0.1~ 0.17~ 0.25~ 0.5 \\ \hline
       Leading track jet $\tau_3$              & 0~ 0.025~ 0.05~ 0.1~ 0.2~ 0.3 \\ \hline
       Subleading track jet $\tau_3$           & 0~ 0.025~ 0.08~ 0.14~ 0.2~ 0.3 \\ \hline

       \end{tabular}
       \caption{The chosen bins for each of the 24 variables being unfolded.}
       \label{tab:IBUBins}
   \end{table}

%Our paper will contain a small number of unfolded histograms to illustrate the breadth of the result.
%We will show plots that show the spectra of the multifold observables, using IBU, MultiFold, and OmniFold in the support note (and also include the correlation dimension~\cite{Komiske:2019fks}) and then we will discuss with the Editorial Board which plots to go in the paper.
\item
  Make public a simulated MC sample (truth-only) that contain a vector of weights for each event: both a nominal MultiFold weight that corrects the MC sample to match the data (i.e.\ provides the central value), and a long list of weight each corresponding to either a systematic or statistical uncertainty source\footnote{This will include the boostrap weights - see Sec.~\ref{sec:uncertaintyuncertainty}.}. We will also make public a Rivet routine that reproduce the event selection, calculates the observables and fills histograms to make prediction with the correct binning.
\end{itemize}

\subsection{Planned content of second paper}
For a second, future paper, the goal is to use true OmniFold and unfold the four momenta of the two muons from the $Z$-boson and a list of the four momenta of all charged hadrons in the event.
This means that there is a different number of observables for each event. The main results would hence be a set of four vectors, and jets can be built from them (using any algorithm).
For visualization, a similar (or the same) set of histograms will be created as from the first paper.

%  \todo{This is only for OmniFold, right? We'd need to make a flat n-tuple with 24 branches, I think, which should be stated here}
For this paper, we plan to make public the weight function(s) (as a neural network, saving the model architecture and weights from TensorFlow) and the exact configuration used to generate the nominal MC events.  There will be one function per systematic uncertainty.  This, combined with a Rivet routine of the analysis selection, will ensure that the data can be used and readily and widely reinterpreted.
%(Alternative: we make public our simulated MC sample with a vector of weights for each event)
